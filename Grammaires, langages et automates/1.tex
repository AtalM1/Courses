\documentclass[a4paper,11pt]{article}

\usepackage{bbm}

% Classic packages
\usepackage[utf8]{inputenc}
\usepackage[french]{babel}
\usepackage{listings, float, graphicx, amsmath, hyperref}

% Useful tweaks
\setcounter{secnumdepth}{2}
\newcommand{\sectionbreak}{\clearpage}
\graphicspath{{./img/}}
\DeclareGraphicsExtensions{.png}
\renewcommand*\thesection{\arabic{section}}
\hypersetup{
    colorlinks,
    citecolor=blue,
    filecolor=blue,
    linkcolor=blue,
    urlcolor=blue
}
\renewcommand{\labelitemi}{---}

% Some shortcuts because I'm lazy
\newcommand{\x}[1]{\texttt{#1}}
\newcommand{\p}{\paragraph*{}}
\newcommand{\bi}{\begin{itemize}}
\newcommand{\ei}{\end{itemize}}
\newcommand{\g}[1]{\textbf{#1}}
\newcommand{\s}[1]{\section{#1}}
% Kills the german sharp S, so use with caution
\renewcommand{\ss}[1]{\subsection{#1}}
\newcommand{\sss}[1]{\subsubsection{#1}}
\newcommand{\img}[3][1]{
 \begin{figure}[H]
  \centering
  \includegraphics[scale=#1]{#2}
  \caption{#3}
 \end{figure}}



\begin{document}
\title{Cours $n^o$1 - Rappels}
\author{Jérémie \textsc{Bourdon}}
\date{2012}
\maketitle
\section{Concepts}
 \subsection{Alphabet}
  \subsubsection{notation}
  \p Pierre de base de toute étude, l'alphabet sera noté $\Sigma$, sa
  taille, ou cardinal, sera notée $\mid\Sigma\mid$.
  \subsubsection{exemples}
  \bi
   \item 26 lettres et l'espace;
   \item ASCII, ASCII étendu, Unicode;
   \item en bio informatique : 4 lettres (ADN/ARN), 20 lettres (protéines)
   \item TALN : dictionnaire (les éléments de l'alphabet seront alors des mots)
  \ei
  \subsubsection{considérations sur la complexité}
  \p Considérons le chiffre de César suivant :\\
  \bi
   \item A $\rightarrow$ E;
   \item B $\rightarrow$ F;
   \item C $\rightarrow$ G;
   \item ...;
   \item Z $\rightarrow$ D.
  \ei
  \p En trouver la clef n'est pas difficile: le E apparaît à environ 20\%{} en
  français : une simple analyse statistique donnera immédiatement
  d'excellents résultats. Si l'on considère maintenant le même code mais sur une
  table ASCII étendue, alors, avec les accents et la casse, cette même analyse se
  révèlera plus complexe à mettre en place.
 \subsection{Mot}
 \p Un mot, ou séquence, est une suite finie de symboles de l'alphabet. L'ensemble
 de tous les mots possibles est noté $\Sigma^*$ avec :
 \[
 \Sigma^* = \bigcup^{\infty}_{i = 0} \Sigma^i
 \]
 où $\Sigma^i$ est l'ensemble des mots de longueur $i$.
 \subsection{Langage}
 \p Un langage est un ensemble de mots (potentiellement infini mais dénombrable).
\section{Comptage}
 \p Prenons comme exemple \x{Bonjour} :\\
 \bi
  \item \x{on} est un facteur de \x{Bonjour};
  \item \x{Bon} est un facteur et un préfixe de \x{Bonjour};
  \item \x{jour} est un facteur et un suffixe de \x{Bonjour};
  \item \x{Bjr} est une sous-séquence de \x{Bonjour} (et n'est pas un facteur).
 \ei
 \p On note $w[i...j]$ le facteur de $w$ composé des symboles de l'indice $i$ à
 l'indice $j$. On note $w[i]$ le $i^{e}$ symbole de $w$.
 \p L'analyse des sous-séquences est utilisée par exemple pour détecter des
 instrusions informatiques : si l'on considère le langage des fichiers de logs
 dont les mots sont une phrase loggée, alors une intrusion se révèle par une
 sous-séquence particulière, d'où la nécessité de les rechercher rapidement.
\section{Distances}
 La notion de distance est assez intuitive, appliquée à deux objets
 dans l'espace, plus la distance est petite plus les objets sont
 proches et lorsque la distance est nulle on peut supposer qu'ils sont
 identiques. On suivra le même raisonnement pour la distance entre
 mots ou entre langages.
 \subsection{Distances entre mots}
  \subsubsection{Distance de Hamming}
  \[
  D_H(w, w') = \sum^{\mid{}w\mid}_{i = 1} \mathbbm{1}_{w[i] \neq w'[i]}
  \]
  \p Il faut en prérequis que $w$ et $w'$ fassent la même longueur.
  Intuitivement, la distance de Hamming est le nombre de caractères qui
  diffèrent d'une chaîne à l'autre.
  \p exemple : $D_H($CHENES$,$ CHEVRE$)=3$
  \subsubsection{Distance de Levenshtein (ou d'édition)}
  \p Soit 3 opérations :\\
  \bi
   \item supprimer un symbole;
   \item insérer un symbole;
   \item modifier un symbole.
  \ei
  \p Alors :
  \[
  D_L(w, w') = \text{nombre minimal d'opérations pour passer de }w\text{ à }w'
  \]
  \p exemple : $D_L($CHENES$,$ CHEVRE$)=3$
  \subsubsection{Extension à la distance de Hamming}
  \p Par un doctorant du LINA : [2002 MANCHERON], STARS.
  \p Cette distance est basée sur une similarité entre séquences :\\ \\
  \begin{tabular}{ l | l l l l }
   $w$  & $w_1$  & $w_2$  & ... & $w_N$  \\
   $w'$ & $w'_1$ & $w'_2$ & ... & $w'_N$ \\ \hline
   $s$  & $s_1$ & $s_2$ & ... & $s_N$ \\
  \end{tabular}
  \p où $w$ et $w'$ sont les mots à comparer et où
  $s_i=\mathbbm{1}_{w[i] = w'[i]}$
  \p $s$ s'écrit :
  \begin{tabular}{ c c c c c c }
   $0...0$  & $1...1$ & $0...0$ & $...$ & $1...1$ & $0...0$ \\
   $z_0$    & $u_1$   & $z_1$   &       & $u_k$   & $z_k$   \\
  \end{tabular}
  \p Deux fonctions, $f_0(n)$ et $f_1(n)$ servent à pondérer les séquences de
  $0$ et de $1$. La similarité s'exprime alors :
  \[
  S(w, w') = \sum^k_{j = 1}\bigg(f_0(z_j) + f_1(u_j)\bigg) + f_0(z_0)
  \]
  \p Pour retomber sur Hamming on prend $f_0(n) = \mid{}n\mid$ et $f_1(n) = 0$.
  \subsubsection{Noyau}
  Soit $F(w)$ l'ensemble des facteurs de $w$. Le noyau de $w$ et $w'$ est
  alors :
  \[
  N(w, w') = F(w) \cap F(w')
  \]
  \p exemple : Soit $w = bon$ et $w' = thon$ alors
  $N(w, w') = \{\epsilon, o, n, on\}$
 \subsection{Distance entre langages}
 On remarquera que sans le support des statistiques, les outils
 permettant de mesurer la distance entre langages sont très peu
 pertinent.
  \subsubsection{Distance de Jaccard}
  Soit $A$ et $B$ deux langages (et donc deux ensembles), alors la distance de
  Jaccard entre ces deux langages est :
  \[
  D_j(A, B) = \frac{\mid A \cup B \mid - \mid A \cap B \mid}{\mid A \cup B \mid}
  \]
  \p exemple : soit $L_1 = \{1, 2, 3, 4, 5, 6, 7, 8, 9\}$ et
  $L_2 = \{6, 7, 8, 9, 10, 11, 12, 13, 14\}$. Alors :
  \[
  D_j(L_1, L_2) = \frac{14 - 4}{14}
  \]
\end{document}
