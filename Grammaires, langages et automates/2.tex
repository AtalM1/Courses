\documentclass[a4paper,11pt]{article}

% Classic packages
\usepackage[utf8]{inputenc}
\usepackage[french]{babel}
\usepackage{listings, float, graphicx, amsmath, hyperref}

% Useful tweaks
\setcounter{secnumdepth}{2}
\newcommand{\sectionbreak}{\clearpage}
\graphicspath{{./img/}}
\DeclareGraphicsExtensions{.png}
\renewcommand*\thesection{\arabic{section}}
\hypersetup{
    colorlinks,
    citecolor=blue,
    filecolor=blue,
    linkcolor=blue,
    urlcolor=blue
}
\renewcommand{\labelitemi}{---}

% Some shortcuts because I'm lazy
\newcommand{\x}[1]{\texttt{#1}}
\newcommand{\p}{\paragraph*{}}
\newcommand{\bi}{\begin{itemize}}
\newcommand{\ei}{\end{itemize}}
\newcommand{\g}[1]{\textbf{#1}}
\newcommand{\s}[1]{\section{#1}}
% Kills the german sharp S, so use with caution
\renewcommand{\ss}[1]{\subsection{#1}}
\newcommand{\sss}[1]{\subsubsection{#1}}
\newcommand{\img}[3][1]{
 \begin{figure}[H]
  \centering
  \includegraphics[scale=#1]{#2}
  \caption{#3}
 \end{figure}}



\begin{document}
\title{Cours $n^o$2 - Langages probabilistes}
\author{Jérémie \textsc{Bourdon}}
\date{2012}
\maketitle
\s{Rappels sur les sources de mots}
 \p Une source de mots est un processus qui génère des suites de symboles sur un
 alphabet fixé (et donc construit des mots).
 \p exemple : un algo qui génère des traces à l'exécution.
 \p Lorsque le processus est aléatoire on parle de source aléatoire de mots.
 \p exemple : pile ou face.
 \p Une source probabiliste quant à elle est une source de mots telle que :
 \[
 \forall n \geq 0, \sum_{w \in \Sigma^n}p_w = 1
 \]
 \p Ces sources de mots définissent une distribution de probabilités sur
 l'ensemble $\Sigma^n$.
 \p Sources classiques : mécanisme fini permettant de modéliser (générer) cette
 distribution.
 \p On peut dessiner ce processus.
\s{Rappels sur les automates}
 \p Un automate est défini par un 5-uplet
 $(\Sigma, \mathcal{Q}, \mathcal{I}, \mathcal{F}, \delta)$ où :\\
 \bi
  \item $\Sigma$ est un alphabet
  \item $\mathcal{Q}$ est un ensemble d'états
  \item $\mathcal{I}$ est un ensemble d'états initiaux
  ($\mathcal{I} \subseteq \mathcal{Q}$)(ou l'état
  initial, cela dépend des définitions, dans ce cas $\mathcal{I} \in \mathcal{Q}$)
  \item $\mathcal{F}$ est un ensemble d'états finaux
  ($\mathcal{F} \subseteq \mathcal{Q}$)
  \item $\delta$ est une fonction de transition définie comme suit :
  \[
  \begin{array}{llll}
   \delta : & \mathcal{Q} \times \Sigma & \to     & \mathcal{Q} \\
            & (q, s)                    & \mapsto & q'\text{ si il y a une
              transition par }s\text{ de }q\text{ à }q'
  \end{array}
  \]
 \ei
  \p Un automate fini déterministe (AFD, ou DFA en anglais) a un nombre fini
  d'états et respecte les contraintes suivantes :
  \bi
   \item il y a un seul état initial
   \item $\forall s \in \Sigma, \forall (q, q', q'') \in \mathcal{Q}^3,
   \delta(q, s) = q' \land \delta(q, s) = q'' \implies q' = q''$
  \ei
  \p Un automate fini non déterministe (AFN, ou NFA en anglais) a un nombre fini
  d'états mais ne respecte pas forcément les contraintes des DFA.
  \p Notons :
  \[
  \begin{array}{l}
  \mathcal{A}_{FD} = \{\text{langages reconnus par un AFD}\} \\
  \mathcal{A}_{FN} = \{\text{langages reconnus par un AFN}\}
  \end{array}
  \]
  \p Alors on a $\mathcal{A}_{FD} = \mathcal{A}_{FN}$.
 
\end{document}
