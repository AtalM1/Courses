\documentclass[a4paper,11pt]{article}

% Classic packages
\usepackage[utf8]{inputenc}
\usepackage[french]{babel}
\usepackage{listings, float, graphicx, amsmath, hyperref}

% Useful tweaks
\setcounter{secnumdepth}{2}
\newcommand{\sectionbreak}{\clearpage}
\graphicspath{{./img/}}
\DeclareGraphicsExtensions{.png}
\renewcommand*\thesection{\arabic{section}}
\hypersetup{
    colorlinks,
    citecolor=blue,
    filecolor=blue,
    linkcolor=blue,
    urlcolor=blue
}
\renewcommand{\labelitemi}{---}

% Some shortcuts because I'm lazy
\newcommand{\x}[1]{\texttt{#1}}
\newcommand{\p}{\paragraph*{}}
\newcommand{\bi}{\begin{itemize}}
\newcommand{\ei}{\end{itemize}}
\newcommand{\g}[1]{\textbf{#1}}
\newcommand{\s}[1]{\section{#1}}
% Kills the german sharp S, so use with caution
\renewcommand{\ss}[1]{\subsection{#1}}
\newcommand{\sss}[1]{\subsubsection{#1}}
\newcommand{\img}[3][1]{
 \begin{figure}[H]
  \centering
  \includegraphics[scale=#1]{#2}
  \caption{#3}
 \end{figure}}



\begin{document}
\title{Cours $n^o$2 - Langages probabilistes}
\author{Jérémie \textsc{Bourdon}}
\date{2012}
\maketitle
\section{Sources de mots aléatoires}
 \p Une source de mots est un processus qui génère des suites de symboles sur un
 alphabet fixé.
 \p exemple : un algo qui génère des traces à l'exécution.
 \p Lorsque le processus est aléatoire on parle de source aléatoire de mots.
 \p exemple : pile ou face.
\end{document}
