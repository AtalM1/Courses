\documentclass[a4paper, 12pt, fleqn]{article}

\usepackage[utf8x]{inputenc}
\usepackage[T1]{fontenc}
\usepackage[francais]{babel}
\usepackage[autolanguage]{numprint}
\usepackage{libertine}
\usepackage{euler}
\usepackage{amsmath}
\usepackage{amssymb}
\usepackage{pgf}
\usepackage{tikz}
\usetikzlibrary{arrows, automata}

\newcommand\K{\mathbb{K}}

\title{Transducteurs Pondérés/Probabilistes}

\author{Jérémie Bourdon}

\date{2012}

\begin{document}

\maketitle

\section{Transducteurs à états finis}
\label{sec:transd-etats-finis}

\subsection{Définitions}
\label{sec:definition}

\paragraph{Transducteur}
\label{sec:transducteur-t}



Un transducteur $T$ est défini par un 6-uplet

\begin{equation*}
  (\Sigma_1, \Sigma_2, Q, q_0, F, \delta)
\end{equation*}
Avec

\begin{equation*}
  \begin{array}{r|l}
    \Sigma_1 & \text{Alphabet d'entrée}         \\
    \Sigma_2 & \text{Alphabet de sortie}        \\
    Q        & \text{Ensemble des états}        \\
    q_0      & \text{État initial}              \\
    F        & \text{Ensemble des états finaux} \\
    \delta   & \text{Fonction de transition}    \\
  \end{array}
\end{equation*}
Et
\begin{equation*}
  \delta: \Sigma_1 \times Q \times \Sigma_2 \rightarrow 2^Q
\end{equation*}

\paragraph{Relation exprimée par un transducteur}
\label{sec:relat-expr-par}

Soit $a = a_1 \dots a_m \in \Sigma_1^*$ et $b = b_1 \dots b_n \in
\Sigma_2^*$. Alors $a$ et $b$ sont en relation si $T$ permet d'écrire
$b$ en lisant $a$ et en s'arr\^etant sur un état final.

$R(T)$ est la relation associée au transducteur $T$.

\subsection{Composition sur les transducteurs}
\label{sec:operations-de-base}

Pour introduire la composition sur les transducteurs, on peut dire que
cela ressemble à faire un pipe sous un système UNIX : la sortie du
premier transducteur est \og branchée\fg{} sur l'entrée du second.

Formellement, soit $T_a (\Sigma_1, \Sigma_2, Q_a, q_{a0}, F_a,
\delta_a)$ et $T_b (\Sigma_2, \Sigma_3, Q_b, q_{b0}, F_b,
\delta_b)$, alors on a :
\begin{equation*}
  R(T_a) \circ R(T_b) := \{(u, w) \in \Sigma_1 \times \Sigma_3 \,
  \vert \, \exists v \in \Sigma_2 \text{ tq } (u, v) \in R(T_a) \text{
    et } (v, w) \in R(T_b)\}
\end{equation*}

Soit $T_1$ un lexique définit comme:\\
\scalebox{0.85}{
  \begin{tikzpicture}[initial text=,%
    initial distance=3mm,%
    node distance=2cm,%
    thick,%
    inner sep=2pt,%
    ->,
    above,
    sloped]
  
    \shorthandoff{:}

    \tikzstyle{every node}=[font=\footnotesize]
  
    \node[initial,state] (1) {}; \node[state] (9) [right of=1] {};
    \node[state] (2) [above of=9] {}; \node[state] (6) [right of=2]
    {}; \node[state] (3) [above of=6] {}; \node[state] (12) [below
    of=9] {};
  
    \foreach \x / \y in {%
      3/{4, 5},%
      6/{7, 8},%
      9/{10, 11},%
      12/{13, ..., 17}} { \def\lastz{\x} \foreach \z [remember=\z as
      \lastz] in \y { \node[state] (\z) [right of=\lastz] {}; } }

    \node[state,accepting] (18) [right of=11, above right of=17] {};
  
    \newcommand\mypath[2]{ \def\lastx{#1} \foreach \x / \y
      [remember=\x as \lastx] in #2 { \path (\lastx) edge [->] node
        {\y:$\varepsilon$} (\x); }}
    \foreach \start / \waypoints in {%
      1/{2/A, 3/L, 4/E, 5/X},%
      2/{6/I, 7/M, 8/E},%
      1/{9/L, 10/E, 11/S},%
      1/{12/F, 13/R, 14/I, 15/T, 16/E, 17/S}}
    \mypath{\start}{\waypoints};
    \foreach \x / \y in {5/ALEX, 8/AIME, 11/LES, 17/FRITES}
      \path (\x) edge node {\textvisiblespace:\y} (18);
    \path (18) edge [bend left=90, looseness=0.65] node {$\varepsilon$:$\varepsilon$} (1);
  \end{tikzpicture}
}

Soit $T_2$, un dictionnaire définit comme:\\
\scalebox{0.85}{
  \begin{tikzpicture}[%
    initial text=, initial distance=3mm,%
    thick, ->,%
    sloped, above]

    \shorthandoff{:}
    
    \tikzstyle{every node}=[font=\footnotesize]
    
    \node [state, initial]   (S) at (0, 0) {};
    \node [state, accepting] (F) at (8, 0) {};
    \path (S) edge [loop above] node {$\varepsilon$:$\varepsilon$} ()
          (S) edge [bend left=60] node {ALEX:NOM\textvisiblespace} (F)
          (S) edge [bend left=40] node {AIME:VERBE\textvisiblespace} (F)
          (S) edge [bend left=20] node {LES:PRONOM\textvisiblespace} (F)
          (S) edge node {FRITES:NOM\textvisiblespace} (F)
          (F) edge [bend left] node {$\varepsilon$:$\varepsilon$} (S);
  \end{tikzpicture}
}

Alors $T_1 \circ T_2$ est le transducteur qui prend une phrase en
entrée et rend les fonctions grammaticales des mots qu'elle comprend.

\section{Transducteurs pondérés \og à la Mohri\fg}
\label{sec:autom-pond}

\subsection{Pondérations}
\label{sec:ponderations}

En sus des transducteurs habituels, il faut définir quels sont les
poids autorisés pour que ceux-ci soient compatibles avec les
opérations sur les transducteurs. On parle de semi-anneaux.

Quelques exemples :

\begin{equation*}
  \begin{array}{l|lllll}
                    & \text{Ensemble}                      & \oplus       & \otimes & \bar{0} & \bar{1} \\
    \hline
    \text{Booléens} & \{0, 1\}                             & \lor         & \land   & \top    & \bot    \\
    \text{Probas}   & \mathbb{R}^+                         & +            & \times  & 0       & 1       \\
    \text{Log}      & \mathbb{R} \cup \{+\infty, -\infty\} & \oplus_{Log} & +       & +\infty & 0       \\
    \text{Tropical} & \mathbb{R} \cup \{+\infty, -\infty\} & min          & +       & +\infty & 0       \\
  \end{array}
\end{equation*}

\subsection{Définition}
\label{sec:definition-1}

Un transducteur pondéré sur un semi anneau $\K$ est défini par
l'octuplet

\begin{equation*}
  (\Sigma_1, \Sigma_2, Q, q_0, F, \delta, \lambda, \rho)
\end{equation*}
Avec

\begin{equation*}
  \begin{array}{r|l}
    \Sigma_1 & \text{Alphabet d'entrée}         \\
    \Sigma_2 & \text{Alphabet de sortie}        \\
    Q        & \text{Ensemble des états}        \\
    q_0      & \text{État initial}              \\
    F        & \text{Ensemble des états finaux} \\
    \delta   & \text{Fonction de transition}    \\
    \lambda  & \text{Pondérations initiales}    \\
    \rho     & \text{Pondérations finales}      \\
  \end{array}
\end{equation*}

Et

\begin{equation*}
  \begin{array}{llll}
    \delta:  & (\Sigma_1 \cup \{\varepsilon\})
    \times Q
    \times (\Sigma_2 \cup \{\varepsilon\})
    \times \mathbb{K}
             & \rightarrow & 2^Q                      \\
    \lambda: & \{q_0\}     & \rightarrow & \mathbb{K} \\
    \rho:    & F           & \rightarrow & \mathbb{K} \\

  \end{array}
\end{equation*}

\subsection{Exemple --- Composition}
\label{sec:exemple}
Soit $T_1$ avec $\K$ tropical :\\
\begin{tikzpicture}[%
  initial text=, initial where=below, initial distance=3mm,%
  thick, ->,%
  node distance=3cm]% arrows
  
  \shorthandoff{:}
  
  \node [state, initial]   (0) at (0, 0) {0};
  \node [state]            (1) [right of=0] {1};
  \node [state]            (2) [right of=1] {2};
  \node [state with output, accepting] (3) [right of=2] {3 \nodepart{lower} 0.7};
  \path (0) edge [bend left] node [above] {a:b/0.1} (1)
        (1) edge [bend left] node [below] {a:b/0.2} (0)
        (1) edge node [above] {b:b/0.3} (2)
        (1) edge [bend right] node [below] {b:b/0.4} (3)
        (2) edge node [above] {a:b/0.5} (3)
        (3) edge [loop above] node {a:a / 0.6} ();
  
\end{tikzpicture}

$T_2$ avec $\K$ tropical:\\
\begin{tikzpicture}[%
  initial text=, initial where=below, initial distance=3mm,%
  thick, ->,%
  node distance=3cm]% arrows
  
  \shorthandoff{:}
  
  \node [state, initial]   (0) at (0, 0) {0};
  \node [state]            (1) [right of=0] {1};
  \node [state]            (2) [right of=1] {2};
  \node [state with output, accepting] (3) [right of=2] {3 \nodepart{lower} 0.6};
  \path (0) edge node [above] {b:b/0.1} (1)
        (1) edge [loop above] node [above] {b:a/0.2} ()
        (1) edge node [above] {a:b/0.3} (2)
        (1) edge [bend right] node [below] {a:b/0.4} (3)
        (2) edge node [above] {b:a/0.5} (3);
  
\end{tikzpicture}

Alors $T_1 \circ T_2$ est:\\
\begin{tikzpicture}[%
  initial text=, initial where=below, initial distance=3mm,%
  thick, ->,%
  node distance=3cm]% arrows
  
  \shorthandoff{:}
  
  \node [state, initial]               (00)               {(0, 0)};
  \node [state]                        (11) [right of=00] {(1, 1)};
  \node [state]                        (01) [above of=11] {(0, 1)};
  \node [state]                        (21) [right of=11] {(2, 1)};
  \node [state]                        (31) [right of=21] {(3, 1)};
  \node [state]                        (32) [above right of=31] {(3, 2)};
  \node [state with output, accepting] (33) [below right of=31] {(3, 3) \nodepart{lower} 1.3};
  \path (00) edge              node [above]         {a:b/0.2} (11)
        (01) edge [bend left]  node [right]         {a:a/0.3} (11)
        (11) edge [bend left]  node [left]          {a:a/0.4} (01)
        (11) edge              node [above]         {b:a/0.5} (21)
        (11) edge [bend right] node [below]         {b:a/0.6} (31)
        (21) edge              node [above]         {a:a/0.7} (31)
        (31) edge              node [above, sloped] {a:b/0.9} (32)
        (31) edge              node [above, sloped] {a:b/1}   (33);
  
\end{tikzpicture}

En effet, la composition est effectuée de m\^eme manière que sans les
pondérations, sauf qu'il faut savoir gérer les poids:

\begin{equation*}
  [T_1 \circ T_2](x, y) = \bigoplus_{z \in \Sigma_2} (T_1(x, z) \otimes T_2(z, y))
\end{equation*}

\end{document}
