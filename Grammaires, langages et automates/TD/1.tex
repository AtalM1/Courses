\documentclass[a4paper,11pt]{article}

\usepackage{bbm}

% Classic packages
\usepackage[utf8]{inputenc}
\usepackage[french]{babel}
\usepackage{listings, float, graphicx, amsmath, hyperref}

% Useful tweaks
\setcounter{secnumdepth}{2}
\newcommand{\sectionbreak}{\clearpage}
\graphicspath{{./img/}}
\DeclareGraphicsExtensions{.png}
\renewcommand*\thesection{\arabic{section}}
\hypersetup{
    colorlinks,
    citecolor=blue,
    filecolor=blue,
    linkcolor=blue,
    urlcolor=blue
}
\renewcommand{\labelitemi}{---}

% Some shortcuts because I'm lazy
\newcommand{\x}[1]{\texttt{#1}}
\newcommand{\p}{\paragraph*{}}
\newcommand{\bi}{\begin{itemize}}
\newcommand{\ei}{\end{itemize}}
\newcommand{\g}[1]{\textbf{#1}}
\newcommand{\s}[1]{\section{#1}}
% Kills the german sharp S, so use with caution
\renewcommand{\ss}[1]{\subsection{#1}}
\newcommand{\sss}[1]{\subsubsection{#1}}
\newcommand{\img}[3][1]{
 \begin{figure}[H]
  \centering
  \includegraphics[scale=#1]{#2}
  \caption{#3}
 \end{figure}}



\begin{document}
\title{TD $n^o$1 - Motifs et distances}
\date{2012}
\maketitle
\s{Exercice 2}
  \paragraph{Question 1} Les compter intuitivement est non trivial, par exemple,
  en marquant d'un tiret chaque lettre utilisée dans la combinaison, on a :\\ \\
  \begin{tabular}{c c c c c c c c c c c c c c c c c c}
  \x{A} & \x{B} & \x{R} & \x{A} & \x{C} & \x{A} & \x{D} &
  \x{A} & \x{B} & \x{R} & \x{A} & \x{C} & \x{A} & \x{D} &
  \x{A} & \x{B} & \x{R} & \x{A} \\
  - & - &   &   &   &   &   &   & - &   & - &   &   &   &   &   &   &   \\ \hline
  - & - &   &   &   &   &   &   & - &   &   &   & - &   &   &   &   &   \\ \hline
  - & - &   &   &   &   &   &   & - &   &   &   &   &   & - &   &   &   \\ \hline
  - & - &   &   &   &   &   &   & - &   &   &   &   &   &   &   &   & - \\ \hline
  - & - &   &   &   &   &   &   &   &   &   &   &   &   &   & - &   & - \\ \hline
  - &   &   &   &   &   &   &   & - &   &   &   &   &   &   & - &   & - \\ \hline
    &   &   & - &   &   &   &   & - &   &   &   &   &   &   & - &   & - \\ \hline
    &   &   &   &   & - &   &   & - &   &   &   &   &   &   & - &   & - \\ \hline
    &   &   &   &   &   &   & - & - &   &   &   &   &   &   & - &   & - \\
  \end{tabular}
  \p Soit 9 occurrences en tout.
  \paragraph{Question 2} La première ligne se calcule facilement (on part de 0
  et on ajoute 1 à chaque fois que l'on croise un \x{A}). Les lignes suivantes
  peuvent se calculer en fonction des valeurs précédentes comme détaillé
  dans la question suivante.\\ \\
  \begin{tabular}{|c|c c c c c c c c c c c c c c c c c c|} \hline
  $_j\backslash ^i$ & \x{A} & \x{B} & \x{R} & \x{A} & \x{C} & \x{A} & \x{D} &
                      \x{A} & \x{B} & \x{R} & \x{A} & \x{C} & \x{A} & \x{D} &
                      \x{A} & \x{B} & \x{R} & \x{A} \\ \hline
  \x{A} & 1 & 1 & 1 & 2 & 2 & 3 & 3 & 4 & 4 & 4 & 5 & 5 & 6 & 6 & 7 & 7 & 7 & 8 \\
  \x{B} & 0 & 1 & 1 & 1 & 1 & 1 & 1 & 1 & 5 & 5 & 5 & 5 & 5 & 5 & 5 & 12& 12& 12\\
  \x{B} & 0 & 0 & 0 & 0 & 0 & 0 & 0 & 0 & 1 & 1 & 1 & 1 & 1 & 1 & 1 & 6 & 6 & 6 \\
  \x{A} & 0 & 0 & 0 & 0 & 0 & 0 & 0 & 0 & 0 & 0 & 1 & 1 & 2 & 2 & 3 & 3 & 3 & 9 \\ \hline
  \end{tabular}
  \paragraph{Question 3} On peut constater assez rapidement ici que pour trouver
  la valeur d'une case, il convient d'additionner la valeur de sa voisine
  de gauche avec celle de sa voisine d'en haut à gauche quand les lettres du
  motif et du texte coïncident, ou simplement de reporter la valeur de sa
  voisine de gauche quand les lettres du motif et du texte ne coïncident pas.
  Autrement dit :
  \[
  M(i, j) =
  \begin{cases}
    M(i - 1, j - 1) + M(i - 1, j) & \quad \text{si }T[i]=w[j]\\
    M(i - 1, j)                   & \quad \text{sinon}
  \end{cases}
  \]
  \paragraph{Question 4} Étant donné que l'on remplit une matrice  de taille
  $\mid T \mid \times \mid w \mid$ pour trouver la solution, la complexité de
  cet algorithme est $\mathcal{O}(\mid T \mid \times \mid w \mid)$.
  \paragraph{Question 5} Il faudrait, au lieu de stocker le nombre
  d'occurrences de telle sous-séquence dans chaque préfixe du texte, stocker la
  position de chacune des lettres composant ces occurrences dans un ensemble
  afin de pouvoir les restituer à la fin.
  \p Cela donnerait, pour commencer (avec le tiret signifiant valeur identique
  à la case de gauche):\\ \\
  \begin{tabular}{|c|c c c c c c c |} \hline
  $_j\backslash ^i$ & \x{A} & \x{B} & \x{R} & \x{A} & \x{C} & \x{A} & $\cdots$ \\ \hline
  \x{A}
   & $\{1\}$
    & -
     & -
      & $\{1, 4\}$
       & -
        & $\{1, 4, 6\}$
         & $\cdots$ \\
  \x{B}
   & $\emptyset$
    & $\{(1,2)\}$
     & -
      & -
       & -
        & -
         & $\cdots$ \\
  \x{B}
   & $\emptyset$
    & -
     & -
      & -
       & -
        & -
         & $\cdots$ \\
  \x{A}
   & $\emptyset$
    & -
     & -
      & -
       & -
        & -
         & $\cdots$ \\ \hline
  \end{tabular}
  \p Il est important de noter que la complexité de l'algorithme explose. En effet,
  le nombre d'occurrences de sous-séquences $w$ dans un texte $T$ peut atteindre
  $2^{\mid w \mid}$ : le nombre de sous-séquences de taille $k$ d'un
  mot de taille $n$ est analogue au nombre de parties de taille $k$ d'un
  ensemble de cardinal $n$. Ce nombre est $\binom{n}{k}$.
  \p Or si on prend $w' = \x{aaaaa}$ et $T' = w'$, on calcule toutes les
  sous-séquences de taille inférieure ou égale à celles qui nous intéressent. On
  se retrouve donc avec
  $\sum_{i = 1}^{\mid w' \mid}\binom{\mid T' \mid}{\mid w' \mid}$
  sous-séquences, ou $2^{\mid w' \mid} - 1$ (nombre de parties d'un ensemble ou
  formule du binôme).
  \p L'explosion de la complexité est donc même exponentielle.
\s{Exercice 4}
  \paragraph{Question 1} Soit $w = \x{CHIEN}$ et $w' = \x{NICHE}$, la matrice
  des $d(i, j)$ est :
  \\ \\
  \begin{tabular}{|c|cccccc|}
  \hline
  $_j\backslash ^i$ & $\epsilon$ & \x{C} & \x{H} & \x{I} & \x{E} & \x{N} \\
  \hline
  $\epsilon$        & 0          & 1     & 2     & 3     & 4     & 5     \\
  \x{N}             & 1          & 1     & 2     & 3     & 4     & 4     \\
  \x{I}             & 2          & 2     & 2     & 2     & 3     & 4     \\
  \x{C}             & 3          & 2     & 3     & 3     & 3     & 4     \\
  \x{H}             & 4          & 3     & 2     & 3     & 4     & 5     \\
  \x{E}             & 5          & 4     & 3     & 3     & 3     & 4     \\
  \hline
  \end{tabular}
  \p Pour répondre plus précisément à la question, on a donc
  $\forall j, d(0, j) = j$ et $\forall i, d(i, 0) = i$.
  \paragraph{Question 2} Posons $ajout = d(i - 1, j) + 1$,
  $suppression = d(i, j - 1) + 1$ et
  $modification = d(i - 1, j - 1) + \mathbbm{1}_{w[i]\neq w'[j]}$. Alors on a :
  \[
  d(i, j) = min(suppression, ajout, modification)
  \]
  \paragraph{Question 3} Un algorithme efficace remplit simplement la matrice
  des $d(i, j)$ en combinant la question 1 pour le cas de base et la question 2
  pour l'hérédité et retourne la valeur $d(\mid w \mid, \mid w' \mid)$ comme
  résultat.
  \paragraph{Question 4} De la même manière que pour la question 4 de l'exercice
  2, la complexité de cet algorithme est
  $\mathcal{O}(\mid w \mid \times \mid w' \mid)$
\end{document}
