\documentclass[a4paper,11pt]{article}

\usepackage{bbm}

% Classic packages
\usepackage[utf8]{inputenc}
\usepackage[french]{babel}
\usepackage{listings, float, graphicx, amsmath, hyperref}

% Useful tweaks
\setcounter{secnumdepth}{2}
\newcommand{\sectionbreak}{\clearpage}
\graphicspath{{./img/}}
\DeclareGraphicsExtensions{.png}
\renewcommand*\thesection{\arabic{section}}
\hypersetup{
    colorlinks,
    citecolor=blue,
    filecolor=blue,
    linkcolor=blue,
    urlcolor=blue
}
\renewcommand{\labelitemi}{---}

% Some shortcuts because I'm lazy
\newcommand{\x}[1]{\texttt{#1}}
\newcommand{\p}{\paragraph*{}}
\newcommand{\bi}{\begin{itemize}}
\newcommand{\ei}{\end{itemize}}
\newcommand{\g}[1]{\textbf{#1}}
\newcommand{\s}[1]{\section{#1}}
% Kills the german sharp S, so use with caution
\renewcommand{\ss}[1]{\subsection{#1}}
\newcommand{\sss}[1]{\subsubsection{#1}}
\newcommand{\img}[3][1]{
 \begin{figure}[H]
  \centering
  \includegraphics[scale=#1]{#2}
  \caption{#3}
 \end{figure}}



\begin{document}
\title{TD $n^o$1 - Motifs et distances}
\date{2012}
\maketitle
\s{Exercice 1}
  \paragraph{Question 1}
  La complexité de l'algorithme naïf est $\mathcal{O}(nm)$
  (deux boucles imbriquées). Si l'on considère que $m$ est fixé, cette
  complexité est simplement $\mathcal{O}(n)$.
  \paragraph{Question 2}
   \subparagraph{a} On peut écrire la définition de l'ensemble
   d'auto-corrélations sous la forme suivante
   \[
   \mathcal{C} = \{v / u, u', v \in \Sigma^*, w = uv \text{ et } w = vu'\}
   \]
   \p Dans cet exercice cet ensemble vaut pour $w$
   $\{\epsilon, A, ABRA, ABRACADABRA\}$
   \subparagraph{b} Pour cette question et la suite de l'exercice (fin de la
   question 2 et questions 3 4 5), voir chapitre 10 du livre {\em Éléments
   d'algorithmique}.
\s{Exercice 2}
  \paragraph{Question 1} Les compter intuitivement est non trivial, par exemple,
  en marquant d'un tiret chaque lettre utilisée dans la combinaison, on a :\\ \\
  \begin{tabular}{*{18}{c}}
  \x{A}&\x{B}&\x{R}&\x{A}&\x{C}&\x{A}&\x{D}&\x{A}&\x{B}&\x{R}&\x{A}&\x{C}&\x{A}&
  \x{D}&\x{A}&\x{B}&\x{R}&\x{A}\\ \hline \hline
  -&-&& && && &-&&-&& && & && \\ \hline
  -&-&& && && &-&& &&-&& & && \\ \hline
  -&-&& && && &-&& && &&-& && \\ \hline
  -&-&& && && &-&& && && & &&-\\ \hline
  -&-&& && && & && && && &-&&-\\ \hline
  -& && && && &-&& && && &-&&-\\ \hline
   & &&-&& && &-&& && && &-&&-\\ \hline
   & && &&-&& &-&& && && &-&&-\\ \hline
   & && && &&-&-&& && && &-&&-\\
  \end{tabular}
  \p Soit 9 occurrences en tout.
  \paragraph{Question 2} La première ligne se calcule facilement (on part de 0
  et on ajoute 1 à chaque fois que l'on croise un \x{A}). Les lignes suivantes
  peuvent se calculer en fonction des valeurs précédentes comme détaillé
  dans la question suivante.\\ \\
  \begin{tabular}{|c|*{18}{c}|} \hline
  $_j\backslash ^i$&\x{A}&\x{B}&\x{R}&\x{A}&\x{C}&\x{A}&\x{D}&\x{A}&\x{B}&\x{R}
                   &\x{A}&\x{C}&\x{A}&\x{D}&\x{A}&\x{B}&\x{R}&\x{A}\\ \hline
  \x{A}&1&1&1&2&2&3&3&4&4&4&5&5&6&6&7&7 &7 & 8 \\
  \x{B}&0&1&1&1&1&1&1&1&5&5&5&5&5&5&5&12&12& 12\\
  \x{B}&0&0&0&0&0&0&0&0&1&1&1&1&1&1&1&6 &6 & 6 \\
  \x{A}&0&0&0&0&0&0&0&0&0&0&1&1&2&2&3&3 &3 & 9 \\ \hline
  \end{tabular}
  \paragraph{Question 3} On peut constater assez rapidement ici que pour trouver
  la valeur d'une case, il convient d'additionner la valeur de sa voisine
  de gauche avec celle de sa voisine d'en haut à gauche quand les lettres du
  motif et du texte coïncident, ou simplement de reporter la valeur de sa
  voisine de gauche quand les lettres du motif et du texte ne coïncident pas.
  Autrement dit :
  \[
  M(i, j) =
  \begin{cases}
    M(i - 1, j - 1) + M(i - 1, j) & \quad \text{si }T[i]=w[j]\\
    M(i - 1, j)                   & \quad \text{sinon}
  \end{cases}
  \]
  \paragraph{Question 4} Étant donné que l'on remplit une matrice  de taille
  $\abs{T} \times \abs{w}$ pour trouver la solution, la complexité de cet
  algorithme est $\mathcal{O}(\abs{T} \times \abs{w})$.
  \paragraph{Question 5} Il faudrait, au lieu de stocker le nombre
  d'occurrences de telle sous-séquence dans chaque préfixe du texte, stocker la
  position de chacune des lettres composant ces occurrences dans un ensemble
  afin de pouvoir les restituer à la fin.
  \p Cela donnerait, pour commencer (avec le tiret signifiant valeur identique
  à la case de gauche):\\ \\
  \begin{tabular}{|c|*{6}{c}|}
  \hline
  $_j\backslash ^i$&\x{A}      &\x{B}      &\x{R}&\x{A}     &\x{C}&\x{A}
  \\ \hline
  \x{A}            &$\{1\}$    &-          &-    &$\{1, 4\}$&-    &$\{1, 4, 6\}$
  \\
  \x{B}            &$\emptyset$&$\{(1,2)\}$&-    &-         &-    &-
  \\
  \x{B}            &$\emptyset$&-          &-    &-         &-    &-
  \\
  \x{A}            &$\emptyset$&-          &-    &-         &-    &-
  \\ \hline
  \end{tabular}
  \p Il est important de noter que la complexité de l'algorithme explose. Pour
  le constater, prenons $a \in \Sigma, w = a^m$ et $T = a^n$. Alors le nombre
  de sous-séquences présentes pour un élément $M(i, j)$ de la matrice $M$ de
  calcul est $\binom{i}{j}$ (car toutes les sous-séquences possibles de taille
  correcte sont retenues, ce qui revient donc à choisir $j$ caractères parmi $i$
  disponibles).
  \p Ensuite, on peut considérer que le nombre de sous-séquences calculées est
  \[
  \sum_{i = 1}^n\left(\sum^m_{j = 1}\binom{i}{j}\right)
  \]
  \p Ou, si l'on admet une optimisation basique (ne pas recalculer les séquences
  déjà calculées précédemment sur une ligne)
  \[
  \sum^m_{j = 1}\binom{n}{j}
  \]
  \p Or
  \[
  \begin{split}
  \binom{n}{m} & = \frac{n!}{(n - m)!m!}\\
               &                        \\
               & = \frac
                    {\overbrace{n(n - 1)...(n - m + 1)}^
                     {\text{m facteurs avoisinant n}}}
                    {\underbrace{m!}_
                     {\text{une constante quand }m\text{ est fixé}}}
  \end{split}
  \]
  \p Donc sans même tenir compte des deux boucles imbriquées et d'après cette
  dernière expression, on voit bien que la complexité temporelle
  va être de $\mathcal{O}(n^m)$ (la boucle de taille fixe ne
  modifie pas cette valeur).
  \p Si l'on suppose qu'aucun choix judicieux n'est fait pour la structure de
  données, voici le nombre de séquences que l'on stockera
  \[
  \sum_{i = 1}^n\left(j\sum^m_{j = 1}\binom{i}{j}\right)
  \]
  (on multiplie par j car une sous-séquence est représentée par j positions de
  lettres)
  \p Ce nombre résulte en la complexité spatiale suivante (pour la même raison
  que la complexité temporelle sauf que cette fois la boucle extérieure est
  prise en compte) : $\mathcal{O}(n^{m + 1})$

\s{Exercice 4}
  \paragraph{Question 1} Soit $w = \x{CHIEN}$ et $w' = \x{NICHE}$, la matrice
  des $d(i, j)$ est :
  \\ \\
  \begin{tabular}{|c|*{6}{c}|}
  \hline
  $_j\backslash ^i$&$\epsilon$&\x{C}&\x{H}&\x{I}&\x{E}&\x{N}\\
  \hline
  $\epsilon$       &0         &1    &2    &3    &4    &5    \\
  \x{N}            &1         &1    &2    &3    &4    &4    \\
  \x{I}            &2         &2    &2    &2    &3    &4    \\
  \x{C}            &3         &2    &3    &3    &3    &4    \\
  \x{H}            &4         &3    &2    &3    &4    &5    \\
  \x{E}            &5         &4    &3    &3    &3    &4    \\
  \hline
  \end{tabular}
  \p Pour répondre plus précisément à la question, on a donc
  $\forall j, d(0, j) = j$ et $\forall i, d(i, 0) = i$.
  \paragraph{Question 2} Posons $ajout = d(i - 1, j) + 1$,
  $suppression = d(i, j - 1) + 1$ et
  $modification = d(i - 1, j - 1) + \mathbbm{1}_{w[i]\neq w'[j]}$. Alors on a :
  \[
  d(i, j) = min(suppression, ajout, modification)
  \]
  \paragraph{Question 3} Un algorithme efficace remplit simplement la matrice
  des $d(i, j)$ en combinant la question 1 pour le cas de base et la question 2
  pour l'hérédité et retourne la valeur $d(\abs{w}, \abs{w'})$ comme
  résultat.
  \paragraph{Question 4} De la même manière que pour la question 4 de l'exercice
  2, la complexité de cet algorithme est
  $\mathcal{O}(\abs{w} \times \abs{w'})$
\end{document}
